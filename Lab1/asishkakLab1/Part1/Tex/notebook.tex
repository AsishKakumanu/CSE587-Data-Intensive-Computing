
% Default to the notebook output style

    


% Inherit from the specified cell style.




    
\documentclass[11pt]{article}

    
    
    \usepackage[T1]{fontenc}
    % Nicer default font (+ math font) than Computer Modern for most use cases
    \usepackage{mathpazo}

    % Basic figure setup, for now with no caption control since it's done
    % automatically by Pandoc (which extracts ![](path) syntax from Markdown).
    \usepackage{graphicx}
    % We will generate all images so they have a width \maxwidth. This means
    % that they will get their normal width if they fit onto the page, but
    % are scaled down if they would overflow the margins.
    \makeatletter
    \def\maxwidth{\ifdim\Gin@nat@width>\linewidth\linewidth
    \else\Gin@nat@width\fi}
    \makeatother
    \let\Oldincludegraphics\includegraphics
    % Set max figure width to be 80% of text width, for now hardcoded.
    \renewcommand{\includegraphics}[1]{\Oldincludegraphics[width=.8\maxwidth]{#1}}
    % Ensure that by default, figures have no caption (until we provide a
    % proper Figure object with a Caption API and a way to capture that
    % in the conversion process - todo).
    \usepackage{caption}
    \DeclareCaptionLabelFormat{nolabel}{}
    \captionsetup{labelformat=nolabel}

    \usepackage{adjustbox} % Used to constrain images to a maximum size 
    \usepackage{xcolor} % Allow colors to be defined
    \usepackage{enumerate} % Needed for markdown enumerations to work
    \usepackage{geometry} % Used to adjust the document margins
    \usepackage{amsmath} % Equations
    \usepackage{amssymb} % Equations
    \usepackage{textcomp} % defines textquotesingle
    % Hack from http://tex.stackexchange.com/a/47451/13684:
    \AtBeginDocument{%
        \def\PYZsq{\textquotesingle}% Upright quotes in Pygmentized code
    }
    \usepackage{upquote} % Upright quotes for verbatim code
    \usepackage{eurosym} % defines \euro
    \usepackage[mathletters]{ucs} % Extended unicode (utf-8) support
    \usepackage[utf8x]{inputenc} % Allow utf-8 characters in the tex document
    \usepackage{fancyvrb} % verbatim replacement that allows latex
    \usepackage{grffile} % extends the file name processing of package graphics 
                         % to support a larger range 
    % The hyperref package gives us a pdf with properly built
    % internal navigation ('pdf bookmarks' for the table of contents,
    % internal cross-reference links, web links for URLs, etc.)
    \usepackage{hyperref}
    \usepackage{longtable} % longtable support required by pandoc >1.10
    \usepackage{booktabs}  % table support for pandoc > 1.12.2
    \usepackage[inline]{enumitem} % IRkernel/repr support (it uses the enumerate* environment)
    \usepackage[normalem]{ulem} % ulem is needed to support strikethroughs (\sout)
                                % normalem makes italics be italics, not underlines
    

    
    
    % Colors for the hyperref package
    \definecolor{urlcolor}{rgb}{0,.145,.698}
    \definecolor{linkcolor}{rgb}{.71,0.21,0.01}
    \definecolor{citecolor}{rgb}{.12,.54,.11}

    % ANSI colors
    \definecolor{ansi-black}{HTML}{3E424D}
    \definecolor{ansi-black-intense}{HTML}{282C36}
    \definecolor{ansi-red}{HTML}{E75C58}
    \definecolor{ansi-red-intense}{HTML}{B22B31}
    \definecolor{ansi-green}{HTML}{00A250}
    \definecolor{ansi-green-intense}{HTML}{007427}
    \definecolor{ansi-yellow}{HTML}{DDB62B}
    \definecolor{ansi-yellow-intense}{HTML}{B27D12}
    \definecolor{ansi-blue}{HTML}{208FFB}
    \definecolor{ansi-blue-intense}{HTML}{0065CA}
    \definecolor{ansi-magenta}{HTML}{D160C4}
    \definecolor{ansi-magenta-intense}{HTML}{A03196}
    \definecolor{ansi-cyan}{HTML}{60C6C8}
    \definecolor{ansi-cyan-intense}{HTML}{258F8F}
    \definecolor{ansi-white}{HTML}{C5C1B4}
    \definecolor{ansi-white-intense}{HTML}{A1A6B2}

    % commands and environments needed by pandoc snippets
    % extracted from the output of `pandoc -s`
    \providecommand{\tightlist}{%
      \setlength{\itemsep}{0pt}\setlength{\parskip}{0pt}}
    \DefineVerbatimEnvironment{Highlighting}{Verbatim}{commandchars=\\\{\}}
    % Add ',fontsize=\small' for more characters per line
    \newenvironment{Shaded}{}{}
    \newcommand{\KeywordTok}[1]{\textcolor[rgb]{0.00,0.44,0.13}{\textbf{{#1}}}}
    \newcommand{\DataTypeTok}[1]{\textcolor[rgb]{0.56,0.13,0.00}{{#1}}}
    \newcommand{\DecValTok}[1]{\textcolor[rgb]{0.25,0.63,0.44}{{#1}}}
    \newcommand{\BaseNTok}[1]{\textcolor[rgb]{0.25,0.63,0.44}{{#1}}}
    \newcommand{\FloatTok}[1]{\textcolor[rgb]{0.25,0.63,0.44}{{#1}}}
    \newcommand{\CharTok}[1]{\textcolor[rgb]{0.25,0.44,0.63}{{#1}}}
    \newcommand{\StringTok}[1]{\textcolor[rgb]{0.25,0.44,0.63}{{#1}}}
    \newcommand{\CommentTok}[1]{\textcolor[rgb]{0.38,0.63,0.69}{\textit{{#1}}}}
    \newcommand{\OtherTok}[1]{\textcolor[rgb]{0.00,0.44,0.13}{{#1}}}
    \newcommand{\AlertTok}[1]{\textcolor[rgb]{1.00,0.00,0.00}{\textbf{{#1}}}}
    \newcommand{\FunctionTok}[1]{\textcolor[rgb]{0.02,0.16,0.49}{{#1}}}
    \newcommand{\RegionMarkerTok}[1]{{#1}}
    \newcommand{\ErrorTok}[1]{\textcolor[rgb]{1.00,0.00,0.00}{\textbf{{#1}}}}
    \newcommand{\NormalTok}[1]{{#1}}
    
    % Additional commands for more recent versions of Pandoc
    \newcommand{\ConstantTok}[1]{\textcolor[rgb]{0.53,0.00,0.00}{{#1}}}
    \newcommand{\SpecialCharTok}[1]{\textcolor[rgb]{0.25,0.44,0.63}{{#1}}}
    \newcommand{\VerbatimStringTok}[1]{\textcolor[rgb]{0.25,0.44,0.63}{{#1}}}
    \newcommand{\SpecialStringTok}[1]{\textcolor[rgb]{0.73,0.40,0.53}{{#1}}}
    \newcommand{\ImportTok}[1]{{#1}}
    \newcommand{\DocumentationTok}[1]{\textcolor[rgb]{0.73,0.13,0.13}{\textit{{#1}}}}
    \newcommand{\AnnotationTok}[1]{\textcolor[rgb]{0.38,0.63,0.69}{\textbf{\textit{{#1}}}}}
    \newcommand{\CommentVarTok}[1]{\textcolor[rgb]{0.38,0.63,0.69}{\textbf{\textit{{#1}}}}}
    \newcommand{\VariableTok}[1]{\textcolor[rgb]{0.10,0.09,0.49}{{#1}}}
    \newcommand{\ControlFlowTok}[1]{\textcolor[rgb]{0.00,0.44,0.13}{\textbf{{#1}}}}
    \newcommand{\OperatorTok}[1]{\textcolor[rgb]{0.40,0.40,0.40}{{#1}}}
    \newcommand{\BuiltInTok}[1]{{#1}}
    \newcommand{\ExtensionTok}[1]{{#1}}
    \newcommand{\PreprocessorTok}[1]{\textcolor[rgb]{0.74,0.48,0.00}{{#1}}}
    \newcommand{\AttributeTok}[1]{\textcolor[rgb]{0.49,0.56,0.16}{{#1}}}
    \newcommand{\InformationTok}[1]{\textcolor[rgb]{0.38,0.63,0.69}{\textbf{\textit{{#1}}}}}
    \newcommand{\WarningTok}[1]{\textcolor[rgb]{0.38,0.63,0.69}{\textbf{\textit{{#1}}}}}
    
    
    % Define a nice break command that doesn't care if a line doesn't already
    % exist.
    \def\br{\hspace*{\fill} \\* }
    % Math Jax compatability definitions
    \def\gt{>}
    \def\lt{<}
    % Document parameters
    \title{kakumanuLab1Part1}
    
    
    

    % Pygments definitions
    
\makeatletter
\def\PY@reset{\let\PY@it=\relax \let\PY@bf=\relax%
    \let\PY@ul=\relax \let\PY@tc=\relax%
    \let\PY@bc=\relax \let\PY@ff=\relax}
\def\PY@tok#1{\csname PY@tok@#1\endcsname}
\def\PY@toks#1+{\ifx\relax#1\empty\else%
    \PY@tok{#1}\expandafter\PY@toks\fi}
\def\PY@do#1{\PY@bc{\PY@tc{\PY@ul{%
    \PY@it{\PY@bf{\PY@ff{#1}}}}}}}
\def\PY#1#2{\PY@reset\PY@toks#1+\relax+\PY@do{#2}}

\expandafter\def\csname PY@tok@w\endcsname{\def\PY@tc##1{\textcolor[rgb]{0.73,0.73,0.73}{##1}}}
\expandafter\def\csname PY@tok@c\endcsname{\let\PY@it=\textit\def\PY@tc##1{\textcolor[rgb]{0.25,0.50,0.50}{##1}}}
\expandafter\def\csname PY@tok@cp\endcsname{\def\PY@tc##1{\textcolor[rgb]{0.74,0.48,0.00}{##1}}}
\expandafter\def\csname PY@tok@k\endcsname{\let\PY@bf=\textbf\def\PY@tc##1{\textcolor[rgb]{0.00,0.50,0.00}{##1}}}
\expandafter\def\csname PY@tok@kp\endcsname{\def\PY@tc##1{\textcolor[rgb]{0.00,0.50,0.00}{##1}}}
\expandafter\def\csname PY@tok@kt\endcsname{\def\PY@tc##1{\textcolor[rgb]{0.69,0.00,0.25}{##1}}}
\expandafter\def\csname PY@tok@o\endcsname{\def\PY@tc##1{\textcolor[rgb]{0.40,0.40,0.40}{##1}}}
\expandafter\def\csname PY@tok@ow\endcsname{\let\PY@bf=\textbf\def\PY@tc##1{\textcolor[rgb]{0.67,0.13,1.00}{##1}}}
\expandafter\def\csname PY@tok@nb\endcsname{\def\PY@tc##1{\textcolor[rgb]{0.00,0.50,0.00}{##1}}}
\expandafter\def\csname PY@tok@nf\endcsname{\def\PY@tc##1{\textcolor[rgb]{0.00,0.00,1.00}{##1}}}
\expandafter\def\csname PY@tok@nc\endcsname{\let\PY@bf=\textbf\def\PY@tc##1{\textcolor[rgb]{0.00,0.00,1.00}{##1}}}
\expandafter\def\csname PY@tok@nn\endcsname{\let\PY@bf=\textbf\def\PY@tc##1{\textcolor[rgb]{0.00,0.00,1.00}{##1}}}
\expandafter\def\csname PY@tok@ne\endcsname{\let\PY@bf=\textbf\def\PY@tc##1{\textcolor[rgb]{0.82,0.25,0.23}{##1}}}
\expandafter\def\csname PY@tok@nv\endcsname{\def\PY@tc##1{\textcolor[rgb]{0.10,0.09,0.49}{##1}}}
\expandafter\def\csname PY@tok@no\endcsname{\def\PY@tc##1{\textcolor[rgb]{0.53,0.00,0.00}{##1}}}
\expandafter\def\csname PY@tok@nl\endcsname{\def\PY@tc##1{\textcolor[rgb]{0.63,0.63,0.00}{##1}}}
\expandafter\def\csname PY@tok@ni\endcsname{\let\PY@bf=\textbf\def\PY@tc##1{\textcolor[rgb]{0.60,0.60,0.60}{##1}}}
\expandafter\def\csname PY@tok@na\endcsname{\def\PY@tc##1{\textcolor[rgb]{0.49,0.56,0.16}{##1}}}
\expandafter\def\csname PY@tok@nt\endcsname{\let\PY@bf=\textbf\def\PY@tc##1{\textcolor[rgb]{0.00,0.50,0.00}{##1}}}
\expandafter\def\csname PY@tok@nd\endcsname{\def\PY@tc##1{\textcolor[rgb]{0.67,0.13,1.00}{##1}}}
\expandafter\def\csname PY@tok@s\endcsname{\def\PY@tc##1{\textcolor[rgb]{0.73,0.13,0.13}{##1}}}
\expandafter\def\csname PY@tok@sd\endcsname{\let\PY@it=\textit\def\PY@tc##1{\textcolor[rgb]{0.73,0.13,0.13}{##1}}}
\expandafter\def\csname PY@tok@si\endcsname{\let\PY@bf=\textbf\def\PY@tc##1{\textcolor[rgb]{0.73,0.40,0.53}{##1}}}
\expandafter\def\csname PY@tok@se\endcsname{\let\PY@bf=\textbf\def\PY@tc##1{\textcolor[rgb]{0.73,0.40,0.13}{##1}}}
\expandafter\def\csname PY@tok@sr\endcsname{\def\PY@tc##1{\textcolor[rgb]{0.73,0.40,0.53}{##1}}}
\expandafter\def\csname PY@tok@ss\endcsname{\def\PY@tc##1{\textcolor[rgb]{0.10,0.09,0.49}{##1}}}
\expandafter\def\csname PY@tok@sx\endcsname{\def\PY@tc##1{\textcolor[rgb]{0.00,0.50,0.00}{##1}}}
\expandafter\def\csname PY@tok@m\endcsname{\def\PY@tc##1{\textcolor[rgb]{0.40,0.40,0.40}{##1}}}
\expandafter\def\csname PY@tok@gh\endcsname{\let\PY@bf=\textbf\def\PY@tc##1{\textcolor[rgb]{0.00,0.00,0.50}{##1}}}
\expandafter\def\csname PY@tok@gu\endcsname{\let\PY@bf=\textbf\def\PY@tc##1{\textcolor[rgb]{0.50,0.00,0.50}{##1}}}
\expandafter\def\csname PY@tok@gd\endcsname{\def\PY@tc##1{\textcolor[rgb]{0.63,0.00,0.00}{##1}}}
\expandafter\def\csname PY@tok@gi\endcsname{\def\PY@tc##1{\textcolor[rgb]{0.00,0.63,0.00}{##1}}}
\expandafter\def\csname PY@tok@gr\endcsname{\def\PY@tc##1{\textcolor[rgb]{1.00,0.00,0.00}{##1}}}
\expandafter\def\csname PY@tok@ge\endcsname{\let\PY@it=\textit}
\expandafter\def\csname PY@tok@gs\endcsname{\let\PY@bf=\textbf}
\expandafter\def\csname PY@tok@gp\endcsname{\let\PY@bf=\textbf\def\PY@tc##1{\textcolor[rgb]{0.00,0.00,0.50}{##1}}}
\expandafter\def\csname PY@tok@go\endcsname{\def\PY@tc##1{\textcolor[rgb]{0.53,0.53,0.53}{##1}}}
\expandafter\def\csname PY@tok@gt\endcsname{\def\PY@tc##1{\textcolor[rgb]{0.00,0.27,0.87}{##1}}}
\expandafter\def\csname PY@tok@err\endcsname{\def\PY@bc##1{\setlength{\fboxsep}{0pt}\fcolorbox[rgb]{1.00,0.00,0.00}{1,1,1}{\strut ##1}}}
\expandafter\def\csname PY@tok@kc\endcsname{\let\PY@bf=\textbf\def\PY@tc##1{\textcolor[rgb]{0.00,0.50,0.00}{##1}}}
\expandafter\def\csname PY@tok@kd\endcsname{\let\PY@bf=\textbf\def\PY@tc##1{\textcolor[rgb]{0.00,0.50,0.00}{##1}}}
\expandafter\def\csname PY@tok@kn\endcsname{\let\PY@bf=\textbf\def\PY@tc##1{\textcolor[rgb]{0.00,0.50,0.00}{##1}}}
\expandafter\def\csname PY@tok@kr\endcsname{\let\PY@bf=\textbf\def\PY@tc##1{\textcolor[rgb]{0.00,0.50,0.00}{##1}}}
\expandafter\def\csname PY@tok@bp\endcsname{\def\PY@tc##1{\textcolor[rgb]{0.00,0.50,0.00}{##1}}}
\expandafter\def\csname PY@tok@fm\endcsname{\def\PY@tc##1{\textcolor[rgb]{0.00,0.00,1.00}{##1}}}
\expandafter\def\csname PY@tok@vc\endcsname{\def\PY@tc##1{\textcolor[rgb]{0.10,0.09,0.49}{##1}}}
\expandafter\def\csname PY@tok@vg\endcsname{\def\PY@tc##1{\textcolor[rgb]{0.10,0.09,0.49}{##1}}}
\expandafter\def\csname PY@tok@vi\endcsname{\def\PY@tc##1{\textcolor[rgb]{0.10,0.09,0.49}{##1}}}
\expandafter\def\csname PY@tok@vm\endcsname{\def\PY@tc##1{\textcolor[rgb]{0.10,0.09,0.49}{##1}}}
\expandafter\def\csname PY@tok@sa\endcsname{\def\PY@tc##1{\textcolor[rgb]{0.73,0.13,0.13}{##1}}}
\expandafter\def\csname PY@tok@sb\endcsname{\def\PY@tc##1{\textcolor[rgb]{0.73,0.13,0.13}{##1}}}
\expandafter\def\csname PY@tok@sc\endcsname{\def\PY@tc##1{\textcolor[rgb]{0.73,0.13,0.13}{##1}}}
\expandafter\def\csname PY@tok@dl\endcsname{\def\PY@tc##1{\textcolor[rgb]{0.73,0.13,0.13}{##1}}}
\expandafter\def\csname PY@tok@s2\endcsname{\def\PY@tc##1{\textcolor[rgb]{0.73,0.13,0.13}{##1}}}
\expandafter\def\csname PY@tok@sh\endcsname{\def\PY@tc##1{\textcolor[rgb]{0.73,0.13,0.13}{##1}}}
\expandafter\def\csname PY@tok@s1\endcsname{\def\PY@tc##1{\textcolor[rgb]{0.73,0.13,0.13}{##1}}}
\expandafter\def\csname PY@tok@mb\endcsname{\def\PY@tc##1{\textcolor[rgb]{0.40,0.40,0.40}{##1}}}
\expandafter\def\csname PY@tok@mf\endcsname{\def\PY@tc##1{\textcolor[rgb]{0.40,0.40,0.40}{##1}}}
\expandafter\def\csname PY@tok@mh\endcsname{\def\PY@tc##1{\textcolor[rgb]{0.40,0.40,0.40}{##1}}}
\expandafter\def\csname PY@tok@mi\endcsname{\def\PY@tc##1{\textcolor[rgb]{0.40,0.40,0.40}{##1}}}
\expandafter\def\csname PY@tok@il\endcsname{\def\PY@tc##1{\textcolor[rgb]{0.40,0.40,0.40}{##1}}}
\expandafter\def\csname PY@tok@mo\endcsname{\def\PY@tc##1{\textcolor[rgb]{0.40,0.40,0.40}{##1}}}
\expandafter\def\csname PY@tok@ch\endcsname{\let\PY@it=\textit\def\PY@tc##1{\textcolor[rgb]{0.25,0.50,0.50}{##1}}}
\expandafter\def\csname PY@tok@cm\endcsname{\let\PY@it=\textit\def\PY@tc##1{\textcolor[rgb]{0.25,0.50,0.50}{##1}}}
\expandafter\def\csname PY@tok@cpf\endcsname{\let\PY@it=\textit\def\PY@tc##1{\textcolor[rgb]{0.25,0.50,0.50}{##1}}}
\expandafter\def\csname PY@tok@c1\endcsname{\let\PY@it=\textit\def\PY@tc##1{\textcolor[rgb]{0.25,0.50,0.50}{##1}}}
\expandafter\def\csname PY@tok@cs\endcsname{\let\PY@it=\textit\def\PY@tc##1{\textcolor[rgb]{0.25,0.50,0.50}{##1}}}

\def\PYZbs{\char`\\}
\def\PYZus{\char`\_}
\def\PYZob{\char`\{}
\def\PYZcb{\char`\}}
\def\PYZca{\char`\^}
\def\PYZam{\char`\&}
\def\PYZlt{\char`\<}
\def\PYZgt{\char`\>}
\def\PYZsh{\char`\#}
\def\PYZpc{\char`\%}
\def\PYZdl{\char`\$}
\def\PYZhy{\char`\-}
\def\PYZsq{\char`\'}
\def\PYZdq{\char`\"}
\def\PYZti{\char`\~}
% for compatibility with earlier versions
\def\PYZat{@}
\def\PYZlb{[}
\def\PYZrb{]}
\makeatother


    % Exact colors from NB
    \definecolor{incolor}{rgb}{0.0, 0.0, 0.5}
    \definecolor{outcolor}{rgb}{0.545, 0.0, 0.0}



    
    % Prevent overflowing lines due to hard-to-break entities
    \sloppy 
    % Setup hyperref package
    \hypersetup{
      breaklinks=true,  % so long urls are correctly broken across lines
      colorlinks=true,
      urlcolor=urlcolor,
      linkcolor=linkcolor,
      citecolor=citecolor,
      }
    % Slightly bigger margins than the latex defaults
    
    \geometry{verbose,tmargin=1in,bmargin=1in,lmargin=1in,rmargin=1in}
    
    

    \begin{document}
    
    
    \maketitle
    
    

    
    Name : Asish Kakumanu UB Person No. : 50288695 UBIT Name : asishkak

\begin{center}\rule{0.5\linewidth}{\linethickness}\end{center}

Teammate

Name : Swapnika P UB Person No. : 50289464 UBIT Name : swapnika 

    \subsection{Basic R Commands}\label{basic-r-commands}

    \begin{Verbatim}[commandchars=\\\{\}]
{\color{incolor}In [{\color{incolor}2}]:} \PY{c+c1}{\PYZsh{} Creating Variables foo and bar with values 2 and 4 respectively.}
        
        foo \PY{o}{\PYZlt{}\PYZhy{}} \PY{l+m}{2}
        bar \PY{o}{\PYZlt{}\PYZhy{}} \PY{l+m}{4} 
        foo \PY{o}{+} bar
\end{Verbatim}


    6

    
    \begin{Verbatim}[commandchars=\\\{\}]
{\color{incolor}In [{\color{incolor}3}]:} \PY{c+c1}{\PYZsh{} Assigning the resultant value of foo + bar to variable (result).}
        
        result \PY{o}{\PYZlt{}\PYZhy{}} foo \PY{o}{+} bar
        result
\end{Verbatim}


    6

    
    \begin{Verbatim}[commandchars=\\\{\}]
{\color{incolor}In [{\color{incolor}18}]:} \PY{c+c1}{\PYZsh{}\PYZsh{} Vectors (Lists)}
         \PY{c+c1}{\PYZsh{}\PYZsh{} Combine all numbers into a vector and assign them to a variable called list.}
         
         \PY{k+kt}{list} \PY{o}{\PYZlt{}\PYZhy{}} \PY{k+kt}{c}\PY{p}{(}\PY{l+m}{2}\PY{p}{,}\PY{l+m}{4}\PY{p}{,}\PY{l+m}{6}\PY{p}{,}\PY{l+m}{8}\PY{p}{)}
         
         \PY{c+c1}{\PYZsh{} Returns item in list with index 2.}
         \PY{k+kt}{list}\PY{p}{[}\PY{l+m}{2}\PY{p}{]}
\end{Verbatim}


    4

    
    \begin{Verbatim}[commandchars=\\\{\}]
{\color{incolor}In [{\color{incolor}19}]:} \PY{c+c1}{\PYZsh{}\PYZsh{} Returns item in list with index 1.}
         \PY{k+kt}{list}\PY{p}{[}\PY{l+m}{1}\PY{p}{]}
\end{Verbatim}


    2

    
    \begin{Verbatim}[commandchars=\\\{\}]
{\color{incolor}In [{\color{incolor}20}]:} \PY{k+kt}{list}\PY{p}{[}\PY{l+m}{0}\PY{p}{]}
\end{Verbatim}


    

    
    \begin{Verbatim}[commandchars=\\\{\}]
{\color{incolor}In [{\color{incolor}21}]:} \PY{k+kt}{list}\PY{p}{[}\PY{l+m}{5}\PY{p}{]}
\end{Verbatim}


    <NA>

    
    \begin{Verbatim}[commandchars=\\\{\}]
{\color{incolor}In [{\color{incolor}22}]:} \PY{c+c1}{\PYZsh{} Adding a value to the list at index 5.}
         
         \PY{k+kt}{list}\PY{p}{[}\PY{l+m}{5}\PY{p}{]} \PY{o}{\PYZlt{}\PYZhy{}} \PY{l+m}{10}
\end{Verbatim}


    \begin{Verbatim}[commandchars=\\\{\}]
{\color{incolor}In [{\color{incolor}24}]:} \PY{c+c1}{\PYZsh{} Return list. }
         
         \PY{k+kt}{list}
\end{Verbatim}


    \begin{enumerate*}
\item 2
\item 4
\item 6
\item 8
\item 10
\end{enumerate*}


    
    \subsection{Arithmetic Operations}\label{arithmetic-operations}

    \begin{Verbatim}[commandchars=\\\{\}]
{\color{incolor}In [{\color{incolor}25}]:} \PY{l+m}{10} \PY{o}{/} \PY{l+m}{2}
\end{Verbatim}


    5

    
    \begin{Verbatim}[commandchars=\\\{\}]
{\color{incolor}In [{\color{incolor}26}]:} \PY{l+m}{2} \PY{o}{==} \PY{l+m}{0}
\end{Verbatim}


    FALSE

    
    \begin{Verbatim}[commandchars=\\\{\}]
{\color{incolor}In [{\color{incolor}27}]:} \PY{l+m}{10} \PY{o}{\PYZca{}} \PY{l+m}{2}
\end{Verbatim}


    100

    
    \begin{Verbatim}[commandchars=\\\{\}]
{\color{incolor}In [{\color{incolor}28}]:} \PY{l+m}{4} \PY{o}{*} \PY{l+m}{5}
\end{Verbatim}


    20

    
    \begin{Verbatim}[commandchars=\\\{\}]
{\color{incolor}In [{\color{incolor}29}]:} \PY{l+m}{1} \PY{o}{+} \PY{l+m}{6}
\end{Verbatim}


    7

    
    \begin{Verbatim}[commandchars=\\\{\}]
{\color{incolor}In [{\color{incolor}30}]:} \PY{p}{(}\PY{l+m}{2}\PY{l+m}{+2}\PY{p}{)} \PY{o}{==} \PY{l+m}{4}
\end{Verbatim}


    TRUE

    
    \begin{Verbatim}[commandchars=\\\{\}]
{\color{incolor}In [{\color{incolor}31}]:} \PY{n+nb+bp}{T} \PY{o}{==} \PY{k+kc}{TRUE}
\end{Verbatim}


    TRUE

    
    \begin{Verbatim}[commandchars=\\\{\}]
{\color{incolor}In [{\color{incolor}32}]:} \PY{n+nb+bp}{F} \PY{o}{\PYZam{}\PYZam{}} \PY{n+nb+bp}{T}
\end{Verbatim}


    FALSE

    
    \begin{Verbatim}[commandchars=\\\{\}]
{\color{incolor}In [{\color{incolor}33}]:} \PY{n+nb+bp}{F} \PY{o}{||} \PY{k+kc}{TRUE}
\end{Verbatim}


    TRUE

    
    \begin{Verbatim}[commandchars=\\\{\}]
{\color{incolor}In [{\color{incolor}34}]:} vect \PY{o}{=} \PY{k+kt}{c}\PY{p}{(}\PY{l+m}{2}\PY{p}{,}\PY{l+m}{4}\PY{p}{,}\PY{l+m}{6}\PY{p}{,}\PY{l+m}{8}\PY{p}{)}
\end{Verbatim}


    \begin{Verbatim}[commandchars=\\\{\}]
{\color{incolor}In [{\color{incolor}35}]:} vect \PY{o}{*} \PY{l+m}{2}
\end{Verbatim}


    \begin{enumerate*}
\item 4
\item 8
\item 12
\item 16
\end{enumerate*}


    
    \begin{Verbatim}[commandchars=\\\{\}]
{\color{incolor}In [{\color{incolor}36}]:} \PY{k+kp}{names}\PY{p}{(}vect\PY{p}{)} \PY{o}{=} \PY{k+kt}{c}\PY{p}{(}\PY{l+s}{\PYZdq{}}\PY{l+s}{1st\PYZdq{}}\PY{p}{,}\PY{l+s}{\PYZdq{}}\PY{l+s}{2nd\PYZdq{}}\PY{p}{,}\PY{l+s}{\PYZdq{}}\PY{l+s}{3rd\PYZdq{}}\PY{p}{,}\PY{l+s}{\PYZdq{}}\PY{l+s}{4th\PYZdq{}}\PY{p}{)}
\end{Verbatim}


    \begin{Verbatim}[commandchars=\\\{\}]
{\color{incolor}In [{\color{incolor}37}]:} vect
\end{Verbatim}


    \begin{description*}
\item[1st] 2
\item[2nd] 4
\item[3rd] 6
\item[4th] 8
\end{description*}


    
    \begin{Verbatim}[commandchars=\\\{\}]
{\color{incolor}In [{\color{incolor}39}]:} vect\PY{p}{[}\PY{l+s}{\PYZdq{}}\PY{l+s}{2nd\PYZdq{}}\PY{p}{]} \PY{o}{\PYZlt{}\PYZhy{}} \PY{l+m}{20}
\end{Verbatim}


    \begin{Verbatim}[commandchars=\\\{\}]
{\color{incolor}In [{\color{incolor}40}]:} vect
\end{Verbatim}


    \begin{description*}
\item[1st] 2
\item[2nd] 20
\item[3rd] 6
\item[4th] 8
\end{description*}


    
    \begin{Verbatim}[commandchars=\\\{\}]
{\color{incolor}In [{\color{incolor}41}]:} demo\PY{p}{(}graphics\PY{p}{)}
\end{Verbatim}


    \begin{Verbatim}[commandchars=\\\{\}]


	demo(graphics)
	---- \textasciitilde{}\textasciitilde{}\textasciitilde{}\textasciitilde{}\textasciitilde{}\textasciitilde{}\textasciitilde{}\textasciitilde{}

> \#  Copyright (C) 1997-2009 The R Core Team
> 
> require(datasets)

> require(grDevices); require(graphics)

> \#\# Here is some code which illustrates some of the differences between
> \#\# R and S graphics capabilities.  Note that colors are generally specified
> \#\# by a character string name (taken from the X11 rgb.txt file) and that line
> \#\# textures are given similarly.  The parameter "bg" sets the background
> \#\# parameter for the plot and there is also an "fg" parameter which sets
> \#\# the foreground color.
> 
> 
> x <- stats::rnorm(50)

> opar <- par(bg = "white")

> plot(x, ann = FALSE, type = "n")

> abline(h = 0, col = gray(.90))

> lines(x, col = "green4", lty = "dotted")

> points(x, bg = "limegreen", pch = 21)

> title(main = "Simple Use of Color In a Plot",
+       xlab = "Just a Whisper of a Label",
+       col.main = "blue", col.lab = gray(.8),
+       cex.main = 1.2, cex.lab = 1.0, font.main = 4, font.lab = 3)

> \#\# A little color wheel.	 This code just plots equally spaced hues in
> \#\# a pie chart.	If you have a cheap SVGA monitor (like me) you will
> \#\# probably find that numerically equispaced does not mean visually
> \#\# equispaced.  On my display at home, these colors tend to cluster at
> \#\# the RGB primaries.  On the other hand on the SGI Indy at work the
> \#\# effect is near perfect.
> 
> par(bg = "gray")

> pie(rep(1,24), col = rainbow(24), radius = 0.9)

    \end{Verbatim}

    \begin{center}
    \adjustimage{max size={0.9\linewidth}{0.9\paperheight}}{output_26_1.png}
    \end{center}
    { \hspace*{\fill} \\}
    
    \begin{Verbatim}[commandchars=\\\{\}]

> title(main = "A Sample Color Wheel", cex.main = 1.4, font.main = 3)

> title(xlab = "(Use this as a test of monitor linearity)",
+       cex.lab = 0.8, font.lab = 3)

> \#\# We have already confessed to having these.  This is just showing off X11
> \#\# color names (and the example (from the postscript manual) is pretty "cute".
> 
> pie.sales <- c(0.12, 0.3, 0.26, 0.16, 0.04, 0.12)

> names(pie.sales) <- c("Blueberry", "Cherry",
+ 		      "Apple", "Boston Cream", "Other", "Vanilla Cream")

> pie(pie.sales,
+     col = c("purple","violetred1","green3","cornsilk","cyan","white"))

    \end{Verbatim}

    \begin{center}
    \adjustimage{max size={0.9\linewidth}{0.9\paperheight}}{output_26_3.png}
    \end{center}
    { \hspace*{\fill} \\}
    
    \begin{Verbatim}[commandchars=\\\{\}]

> title(main = "January Pie Sales", cex.main = 1.8, font.main = 1)

> title(xlab = "(Don't try this at home kids)", cex.lab = 0.8, font.lab = 3)

> \#\# Boxplots:  I couldn't resist the capability for filling the "box".
> \#\# The use of color seems like a useful addition, it focuses attention
> \#\# on the central bulk of the data.
> 
> par(bg="cornsilk")

> n <- 10

> g <- gl(n, 100, n*100)

> x <- rnorm(n*100) + sqrt(as.numeric(g))

> boxplot(split(x,g), col="lavender", notch=TRUE)

    \end{Verbatim}

    \begin{center}
    \adjustimage{max size={0.9\linewidth}{0.9\paperheight}}{output_26_5.png}
    \end{center}
    { \hspace*{\fill} \\}
    
    \begin{Verbatim}[commandchars=\\\{\}]

> title(main="Notched Boxplots", xlab="Group", font.main=4, font.lab=1)

> \#\# An example showing how to fill between curves.
> 
> par(bg="white")

> n <- 100

> x <- c(0,cumsum(rnorm(n)))

> y <- c(0,cumsum(rnorm(n)))

> xx <- c(0:n, n:0)

> yy <- c(x, rev(y))

> plot(xx, yy, type="n", xlab="Time", ylab="Distance")

    \end{Verbatim}

    \begin{center}
    \adjustimage{max size={0.9\linewidth}{0.9\paperheight}}{output_26_7.png}
    \end{center}
    { \hspace*{\fill} \\}
    
    \begin{Verbatim}[commandchars=\\\{\}]

> polygon(xx, yy, col="gray")

> title("Distance Between Brownian Motions")

> \#\# Colored plot margins, axis labels and titles.	 You do need to be
> \#\# careful with these kinds of effects.	It's easy to go completely
> \#\# over the top and you can end up with your lunch all over the keyboard.
> \#\# On the other hand, my market research clients love it.
> 
> x <- c(0.00, 0.40, 0.86, 0.85, 0.69, 0.48, 0.54, 1.09, 1.11, 1.73, 2.05, 2.02)

> par(bg="lightgray")

> plot(x, type="n", axes=FALSE, ann=FALSE)

    \end{Verbatim}

    \begin{center}
    \adjustimage{max size={0.9\linewidth}{0.9\paperheight}}{output_26_9.png}
    \end{center}
    { \hspace*{\fill} \\}
    
    \begin{Verbatim}[commandchars=\\\{\}]

> usr <- par("usr")

> rect(usr[1], usr[3], usr[2], usr[4], col="cornsilk", border="black")

> lines(x, col="blue")

> points(x, pch=21, bg="lightcyan", cex=1.25)

> axis(2, col.axis="blue", las=1)

> axis(1, at=1:12, lab=month.abb, col.axis="blue")

> box()

> title(main= "The Level of Interest in R", font.main=4, col.main="red")

> title(xlab= "1996", col.lab="red")

> \#\# A filled histogram, showing how to change the font used for the
> \#\# main title without changing the other annotation.
> 
> par(bg="cornsilk")

> x <- rnorm(1000)

> hist(x, xlim=range(-4, 4, x), col="lavender", main="")

    \end{Verbatim}

    \begin{center}
    \adjustimage{max size={0.9\linewidth}{0.9\paperheight}}{output_26_11.png}
    \end{center}
    { \hspace*{\fill} \\}
    
    \begin{Verbatim}[commandchars=\\\{\}]

> title(main="1000 Normal Random Variates", font.main=3)

> \#\# A scatterplot matrix
> \#\# The good old Iris data (yet again)
> 
> pairs(iris[1:4], main="Edgar Anderson's Iris Data", font.main=4, pch=19)

    \end{Verbatim}

    \begin{center}
    \adjustimage{max size={0.9\linewidth}{0.9\paperheight}}{output_26_13.png}
    \end{center}
    { \hspace*{\fill} \\}
    
    \begin{Verbatim}[commandchars=\\\{\}]

> pairs(iris[1:4], main="Edgar Anderson's Iris Data", pch=21,
+       bg = c("red", "green3", "blue")[unclass(iris\$Species)])

    \end{Verbatim}

    \begin{center}
    \adjustimage{max size={0.9\linewidth}{0.9\paperheight}}{output_26_15.png}
    \end{center}
    { \hspace*{\fill} \\}
    
    \begin{Verbatim}[commandchars=\\\{\}]

> \#\# Contour plotting
> \#\# This produces a topographic map of one of Auckland's many volcanic "peaks".
> 
> x <- 10*1:nrow(volcano)

> y <- 10*1:ncol(volcano)

> lev <- pretty(range(volcano), 10)

> par(bg = "lightcyan")

> pin <- par("pin")

> xdelta <- diff(range(x))

> ydelta <- diff(range(y))

> xscale <- pin[1]/xdelta

> yscale <- pin[2]/ydelta

> scale <- min(xscale, yscale)

> xadd <- 0.5*(pin[1]/scale - xdelta)

> yadd <- 0.5*(pin[2]/scale - ydelta)

> plot(numeric(0), numeric(0),
+      xlim = range(x)+c(-1,1)*xadd, ylim = range(y)+c(-1,1)*yadd,
+      type = "n", ann = FALSE)

    \end{Verbatim}

    \begin{center}
    \adjustimage{max size={0.9\linewidth}{0.9\paperheight}}{output_26_17.png}
    \end{center}
    { \hspace*{\fill} \\}
    
    \begin{Verbatim}[commandchars=\\\{\}]

> usr <- par("usr")

> rect(usr[1], usr[3], usr[2], usr[4], col="green3")

> contour(x, y, volcano, levels = lev, col="yellow", lty="solid", add=TRUE)

> box()

> title("A Topographic Map of Maunga Whau", font= 4)

> title(xlab = "Meters North", ylab = "Meters West", font= 3)

> mtext("10 Meter Contour Spacing", side=3, line=0.35, outer=FALSE,
+       at = mean(par("usr")[1:2]), cex=0.7, font=3)

> \#\# Conditioning plots
> 
> par(bg="cornsilk")

> coplot(lat \textasciitilde{} long | depth, data = quakes, pch = 21, bg = "green3")

    \end{Verbatim}

    \begin{center}
    \adjustimage{max size={0.9\linewidth}{0.9\paperheight}}{output_26_19.png}
    \end{center}
    { \hspace*{\fill} \\}
    
    \begin{Verbatim}[commandchars=\\\{\}]

> par(opar)

    \end{Verbatim}

    \begin{center}
    \adjustimage{max size={0.9\linewidth}{0.9\paperheight}}{output_26_21.png}
    \end{center}
    { \hspace*{\fill} \\}
    
    \begin{Verbatim}[commandchars=\\\{\}]
{\color{incolor}In [{\color{incolor}42}]:} \PY{c+c1}{\PYZsh{} A function name generateSquares which prints the square of the given number.}
         
         generateSquares \PY{o}{\PYZlt{}\PYZhy{}} \PY{k+kr}{function}\PY{p}{(}x\PY{p}{)}\PY{p}{\PYZob{}}
             \PY{k+kr}{return}\PY{p}{(}x\PY{o}{\PYZca{}}\PY{l+m}{2}\PY{p}{)}
         \PY{p}{\PYZcb{}}
\end{Verbatim}


    \begin{Verbatim}[commandchars=\\\{\}]
{\color{incolor}In [{\color{incolor}43}]:} generateSquares\PY{p}{(}\PY{l+m}{4}\PY{p}{)}
\end{Verbatim}


    16

    
    \begin{Verbatim}[commandchars=\\\{\}]
{\color{incolor}In [{\color{incolor}44}]:} \PY{c+c1}{\PYZsh{} A function with three arguments a,b,c. }
         
         func\PYZus{}arguments \PY{o}{\PYZlt{}\PYZhy{}} \PY{k+kr}{function} \PY{p}{(}a\PY{o}{=}\PY{l+m}{1}\PY{p}{,}b\PY{o}{=}\PY{l+m}{2}\PY{p}{,}\PY{k+kt}{c}\PY{o}{=}\PY{l+m}{3}\PY{p}{)}\PY{p}{\PYZob{}}
             res \PY{o}{\PYZlt{}\PYZhy{}} a \PY{o}{+} \PY{p}{(}b \PY{o}{*} \PY{k+kt}{c}\PY{p}{)}
             \PY{k+kp}{print}\PY{p}{(}res\PY{p}{)}
         \PY{p}{\PYZcb{}}
\end{Verbatim}


    \begin{Verbatim}[commandchars=\\\{\}]
{\color{incolor}In [{\color{incolor}45}]:} \PY{c+c1}{\PYZsh{} Pass values to arguments with index.}
         
         func\PYZus{}arguments\PY{p}{(}\PY{l+m}{4}\PY{p}{,}\PY{l+m}{5}\PY{p}{,}\PY{l+m}{6}\PY{p}{)}
\end{Verbatim}


    \begin{Verbatim}[commandchars=\\\{\}]
[1] 34

    \end{Verbatim}

    \begin{Verbatim}[commandchars=\\\{\}]
{\color{incolor}In [{\color{incolor}46}]:} \PY{c+c1}{\PYZsh{} Passing values to arguments using name of arguments }
         
         func\PYZus{}arguments\PY{p}{(}a\PY{o}{=}\PY{l+m}{4}\PY{p}{,}b\PY{o}{=}\PY{l+m}{5}\PY{p}{,}\PY{k+kt}{c}\PY{o}{=}\PY{l+m}{2}\PY{p}{)}
\end{Verbatim}


    \begin{Verbatim}[commandchars=\\\{\}]
[1] 14

    \end{Verbatim}

    \begin{Verbatim}[commandchars=\\\{\}]
{\color{incolor}In [{\color{incolor}47}]:} mydataframe \PY{o}{\PYZlt{}\PYZhy{}} \PY{k+kt}{data.frame}\PY{p}{(}
             
         \PY{c+c1}{\PYZsh{} Creates a vector (list) starting from 1 to 5.}
         stu\PYZus{}id \PY{o}{=} \PY{k+kt}{c}\PY{p}{(}\PY{l+m}{1}\PY{o}{:}\PY{l+m}{5}\PY{p}{)}\PY{p}{,}
         \PY{c+c1}{\PYZsh{} Creates a vector (list) using the below names.}
         stu\PYZus{}name \PY{o}{=} \PY{k+kt}{c}\PY{p}{(}\PY{l+s}{\PYZdq{}}\PY{l+s}{Bob\PYZdq{}}\PY{p}{,}\PY{l+s}{\PYZdq{}}\PY{l+s}{Pat\PYZdq{}}\PY{p}{,}\PY{l+s}{\PYZdq{}}\PY{l+s}{Jane\PYZdq{}}\PY{p}{,}\PY{l+s}{\PYZdq{}}\PY{l+s}{Peter\PYZdq{}}\PY{p}{,}\PY{l+s}{\PYZdq{}}\PY{l+s}{Han\PYZdq{}}\PY{p}{)}\PY{p}{,}
         \PY{c+c1}{\PYZsh{} To avoid problems when reassigning values within a dataframe. We use stringsAsFactors.}
         stringsAsFactors \PY{o}{=} \PY{k+kc}{FALSE}
         \PY{p}{)}
\end{Verbatim}


    \begin{Verbatim}[commandchars=\\\{\}]
{\color{incolor}In [{\color{incolor}48}]:} res  \PY{o}{\PYZlt{}\PYZhy{}} \PY{k+kt}{data.frame}\PY{p}{(}mydataframe\PY{o}{\PYZdl{}}stu\PYZus{}id\PY{p}{,}mydataframe\PY{o}{\PYZdl{}}stu\PYZus{}name\PY{p}{)}
\end{Verbatim}


    \begin{Verbatim}[commandchars=\\\{\}]
{\color{incolor}In [{\color{incolor}49}]:} res
\end{Verbatim}


    \begin{tabular}{r|ll}
 mydataframe.stu\_id & mydataframe.stu\_name\\
\hline
	 1     & Bob  \\
	 2     & Pat  \\
	 3     & Jane \\
	 4     & Peter\\
	 5     & Han  \\
\end{tabular}


    
    \subsection{Problem 1}\label{problem-1}

    \begin{Verbatim}[commandchars=\\\{\}]
{\color{incolor}In [{\color{incolor}6}]:} \PY{c+c1}{\PYZsh{} pch \PYZhy{} Generates a Symbol}
        \PY{c+c1}{\PYZsh{} lty \PYZhy{} line type (Solid, Dashled line)}
        \PY{c+c1}{\PYZsh{} col \PYZhy{} Color}
        \PY{c+c1}{\PYZsh{} Inset \PYZhy{} Place to display on the grid}
        \PY{c+c1}{\PYZsh{} rpois \PYZhy{} Generates multinomial or multi\PYZhy{}Poission random variates based on an Aitchison composition}
        \PY{c+c1}{\PYZsh{} nx, ny \PYZhy{} Horizontal and Vertical lines.}
        
        \PY{c+c1}{\PYZsh{} A vector of values assigned to variable \PYZsq{}sales1\PYZsq{}.}
        sales1 \PY{o}{\PYZlt{}\PYZhy{}} \PY{k+kt}{c}\PY{p}{(}\PY{l+m}{12}\PY{p}{,}\PY{l+m}{14}\PY{p}{,}\PY{l+m}{16}\PY{p}{,}\PY{l+m}{29}\PY{p}{,}\PY{l+m}{30}\PY{p}{,}\PY{l+m}{45}\PY{p}{,}\PY{l+m}{19}\PY{p}{,}\PY{l+m}{20}\PY{p}{,}\PY{l+m}{16}\PY{p}{,}\PY{l+m}{19}\PY{p}{,}\PY{l+m}{34}\PY{p}{,}\PY{l+m}{20}\PY{p}{)}
        
        \PY{c+c1}{\PYZsh{} A vector of random values between 12 to 34 are assigned to \PYZsq{}sales2\PYZsq{}.}
        sales2 \PY{o}{\PYZlt{}\PYZhy{}} rpois\PY{p}{(}\PY{l+m}{12}\PY{p}{,}\PY{l+m}{34}\PY{p}{)}
        
        \PY{c+c1}{\PYZsh{} Background Color}
        par\PY{p}{(}bg\PY{o}{=}\PY{l+s}{\PYZdq{}}\PY{l+s}{cornsilk\PYZdq{}}\PY{p}{)}
        
        \PY{c+c1}{\PYZsh{} Plot sales1 with color blue, with each point using \PYZsq{}o\PYZsq{}.}
        \PY{c+c1}{\PYZsh{} y\PYZhy{}axis extends from 0 to 100. }
        \PY{c+c1}{\PYZsh{} xlab \PYZam{} ylab \PYZhy{}\PYZgt{} x label and y label}
        \PY{c+c1}{\PYZsh{} title \PYZhy{} main \PYZhy{}\PYZgt{} Main title}
        \PY{c+c1}{\PYZsh{} title \PYZhy{} sub \PYZhy{}\PYZgt{} Sub title}
        plot\PY{p}{(}sales1\PY{p}{,} col\PY{o}{=}\PY{l+s}{\PYZdq{}}\PY{l+s}{blue\PYZdq{}}\PY{p}{,} type\PY{o}{=}\PY{l+s}{\PYZdq{}}\PY{l+s}{o\PYZdq{}}\PY{p}{,} ylim\PY{o}{=}\PY{k+kt}{c}\PY{p}{(}\PY{l+m}{0}\PY{p}{,}\PY{l+m}{100}\PY{p}{)}\PY{p}{,} xlab\PY{o}{=}\PY{l+s}{\PYZdq{}}\PY{l+s}{Month\PYZdq{}}\PY{p}{,} ylab\PY{o}{=}\PY{l+s}{\PYZdq{}}\PY{l+s}{Sales\PYZdq{}} \PY{p}{)}
        title\PY{p}{(}main\PY{o}{=}\PY{l+s}{\PYZdq{}}\PY{l+s}{Sales by Month\PYZdq{}}\PY{p}{)}
        
        \PY{c+c1}{\PYZsh{} Plot a line using randomly generated data sales2. }
        \PY{c+c1}{\PYZsh{} pch code 22 generates a square. }
        \PY{c+c1}{\PYZsh{} lty \PYZhy{} Line type (Solid or Dashed Line)}
        lines\PY{p}{(}sales2\PY{p}{,} type\PY{o}{=}\PY{l+s}{\PYZdq{}}\PY{l+s}{o\PYZdq{}}\PY{p}{,} pch\PY{o}{=}\PY{l+m}{22}\PY{p}{,} lty\PY{o}{=}\PY{l+m}{2}\PY{p}{,} col\PY{o}{=}\PY{l+s}{\PYZdq{}}\PY{l+s}{red\PYZdq{}}\PY{p}{)}
        grid\PY{p}{(}nx\PY{o}{=}\PY{k+kc}{NA}\PY{p}{,} ny\PY{o}{=}\PY{k+kc}{NULL}\PY{p}{)}
        
        \PY{c+c1}{\PYZsh{} Fix a legend in topright corner of the grid with a margin of .05. }
        \PY{c+c1}{\PYZsh{} Create a vector (sales1, sales2) along with colors set as blue and red respectively.}
        legend\PY{p}{(}\PY{l+s}{\PYZdq{}}\PY{l+s}{topright\PYZdq{}}\PY{p}{,} inset\PY{o}{=}\PY{l+m}{.05}\PY{p}{,} \PY{k+kt}{c}\PY{p}{(}\PY{l+s}{\PYZdq{}}\PY{l+s}{Sales1\PYZdq{}}\PY{p}{,}\PY{l+s}{\PYZdq{}}\PY{l+s}{Sales2\PYZdq{}}\PY{p}{)}\PY{p}{,} fill\PY{o}{=}\PY{k+kt}{c}\PY{p}{(}\PY{l+s}{\PYZdq{}}\PY{l+s}{blue\PYZdq{}}\PY{p}{,}\PY{l+s}{\PYZdq{}}\PY{l+s}{red\PYZdq{}}\PY{p}{)}\PY{p}{,} horiz\PY{o}{=}\PY{k+kc}{TRUE}\PY{p}{)}
\end{Verbatim}


    \begin{center}
    \adjustimage{max size={0.9\linewidth}{0.9\paperheight}}{output_36_0.png}
    \end{center}
    { \hspace*{\fill} \\}
    
    \subsection{Problem 2}\label{problem-2}

    \begin{Verbatim}[commandchars=\\\{\}]
{\color{incolor}In [{\color{incolor}10}]:} \PY{c+c1}{\PYZsh{} read.delim \PYZhy{}\PYZgt{} To read a delimited txt file. Here salesdata.txt is delimited with spaces.}
         \PY{c+c1}{\PYZsh{} sales\PYZlt{}\PYZhy{}read.table(file.choose(), header=T)}
         sales \PY{o}{\PYZlt{}\PYZhy{}} read.delim\PY{p}{(}\PY{l+s}{\PYZsq{}}\PY{l+s}{salesdata.txt\PYZsq{}}\PY{p}{)}
\end{Verbatim}


    \begin{Verbatim}[commandchars=\\\{\}]
{\color{incolor}In [{\color{incolor}11}]:} sales
\end{Verbatim}


    \begin{tabular}{r|ll}
 Sales1 & Sales2\\
\hline
	 34 & 45\\
	 23 & 23\\
	 12 & 56\\
	 29 & 34\\
	 34 & 45\\
	 34 & 56\\
	 12 & 89\\
	 80 & 36\\
	 80 & 44\\
	 56 & 66\\
	 55 & 60\\
	 45 & 23\\
\end{tabular}


    
    \begin{Verbatim}[commandchars=\\\{\}]
{\color{incolor}In [{\color{incolor}13}]:} \PY{c+c1}{\PYZsh{} as.matrix \PYZhy{} \PYZgt{} Returns all values of vector into a matrix}
         \PY{c+c1}{\PYZsh{} barplot \PYZhy{}\PYZgt{} Plot a bar plot using the data}
         \PY{c+c1}{\PYZsh{} beside \PYZhy{}\PYZgt{} When False, they stack horizontally. When True, columns are potrayed as stacked bars.}
         barplot\PY{p}{(}\PY{k+kp}{as.matrix}\PY{p}{(}sales\PY{p}{)}\PY{p}{,} main\PY{o}{=}\PY{l+s}{\PYZdq{}}\PY{l+s}{Sales Data\PYZdq{}}\PY{p}{,} ylab\PY{o}{=} \PY{l+s}{\PYZdq{}}\PY{l+s}{Total\PYZdq{}}\PY{p}{,}beside\PY{o}{=}\PY{n+nb+bp}{T}\PY{p}{,} col\PY{o}{=}rainbow\PY{p}{(}\PY{l+m}{5}\PY{p}{)}\PY{p}{)}
\end{Verbatim}


    \begin{center}
    \adjustimage{max size={0.9\linewidth}{0.9\paperheight}}{output_40_0.png}
    \end{center}
    { \hspace*{\fill} \\}
    
    \subsection{Problem 3}\label{problem-3}

    \begin{Verbatim}[commandchars=\\\{\}]
{\color{incolor}In [{\color{incolor}20}]:} \PY{c+c1}{\PYZsh{} boxplot for sales. Two colors for two columns}
         fn\PY{o}{\PYZlt{}\PYZhy{}}boxplot\PY{p}{(}sales\PY{p}{,}col\PY{o}{=}\PY{k+kt}{c}\PY{p}{(}\PY{l+s}{\PYZdq{}}\PY{l+s}{orange\PYZdq{}}\PY{p}{,}\PY{l+s}{\PYZdq{}}\PY{l+s}{green\PYZdq{}}\PY{p}{)}\PY{p}{)}\PY{o}{\PYZdl{}}stats
         
         \PY{c+c1}{\PYZsh{} fn converts a argument to function}
         \PY{c+c1}{\PYZsh{} text places a text. }
         \PY{c+c1}{\PYZsh{} cex \PYZhy{}\PYZgt{} Character size}
         
         text\PY{p}{(}\PY{l+m}{1.45}\PY{p}{,} fn\PY{p}{[}\PY{l+m}{3}\PY{p}{,}\PY{l+m}{2}\PY{p}{]}\PY{p}{,} \PY{k+kp}{paste}\PY{p}{(}\PY{l+s}{\PYZdq{}}\PY{l+s}{Median =\PYZdq{}}\PY{p}{,} fn\PY{p}{[}\PY{l+m}{3}\PY{p}{,}\PY{l+m}{2}\PY{p}{]}\PY{p}{)}\PY{p}{,} adj\PY{o}{=}\PY{l+m}{0}\PY{p}{,} cex\PY{o}{=}\PY{l+m}{.7}\PY{p}{)}
         text\PY{p}{(}\PY{l+m}{0.45}\PY{p}{,} fn\PY{p}{[}\PY{l+m}{3}\PY{p}{,}\PY{l+m}{1}\PY{p}{]}\PY{p}{,}\PY{k+kp}{paste}\PY{p}{(}\PY{l+s}{\PYZdq{}}\PY{l+s}{Median =\PYZdq{}}\PY{p}{,} fn\PY{p}{[}\PY{l+m}{3}\PY{p}{,}\PY{l+m}{1}\PY{p}{]}\PY{p}{)}\PY{p}{,} adj\PY{o}{=}\PY{l+m}{0}\PY{p}{,} cex\PY{o}{=}\PY{l+m}{.7}\PY{p}{)}
         grid\PY{p}{(}nx\PY{o}{=}\PY{k+kc}{NA}\PY{p}{,} ny\PY{o}{=}\PY{k+kc}{NULL}\PY{p}{)}
\end{Verbatim}


    \begin{center}
    \adjustimage{max size={0.9\linewidth}{0.9\paperheight}}{output_42_0.png}
    \end{center}
    { \hspace*{\fill} \\}
    
    \subsection{Problem 4}\label{problem-4}

    \begin{Verbatim}[commandchars=\\\{\}]
{\color{incolor}In [{\color{incolor}28}]:} \PY{c+c1}{\PYZsh{} Read FB.csv file to fb1 variable}
         \PY{c+c1}{\PYZsh{} Similarly aapl1}
         \PY{c+c1}{\PYZsh{} plot \PYZsq{}adj close\PYZsq{} from appl1 with blue color and with \PYZsq{}o\PYZsq{}.}
         \PY{c+c1}{\PYZsh{} plot \PYZsq{}adj close\PYZsq{} from fb1 with red color and with a symbol \PYZsq{}pch = 22\PYZsq{}.}
         \PY{c+c1}{\PYZsh{} Histogram of column \PYZsq{}Adj Close\PYZsq{} from appl1.}
         
         fb1\PY{o}{\PYZlt{}\PYZhy{}}read.csv\PY{p}{(}\PY{l+s}{\PYZsq{}}\PY{l+s}{FB.csv\PYZsq{}}\PY{p}{)}
         aapl1\PY{o}{\PYZlt{}\PYZhy{}}read.csv\PY{p}{(}\PY{l+s}{\PYZsq{}}\PY{l+s}{AAPL.csv\PYZsq{}}\PY{p}{)}
         par\PY{p}{(}bg\PY{o}{=}\PY{l+s}{\PYZdq{}}\PY{l+s}{cornsilk\PYZdq{}}\PY{p}{)}
         plot\PY{p}{(}aapl1\PY{o}{\PYZdl{}}Adj.Close\PY{p}{,} col\PY{o}{=}\PY{l+s}{\PYZdq{}}\PY{l+s}{blue\PYZdq{}}\PY{p}{,} type\PY{o}{=}\PY{l+s}{\PYZdq{}}\PY{l+s}{o\PYZdq{}}\PY{p}{,} ylim\PY{o}{=}\PY{k+kt}{c}\PY{p}{(}\PY{l+m}{0}\PY{p}{,}\PY{l+m}{400}\PY{p}{)}\PY{p}{,} xlab\PY{o}{=}\PY{l+s}{\PYZdq{}}\PY{l+s}{Days\PYZdq{}}\PY{p}{,} ylab\PY{o}{=}\PY{l+s}{\PYZdq{}}\PY{l+s}{Price\PYZdq{}} \PY{p}{)}
         lines\PY{p}{(}fb1\PY{o}{\PYZdl{}}Adj.Close\PY{p}{,} type\PY{o}{=}\PY{l+s}{\PYZdq{}}\PY{l+s}{o\PYZdq{}}\PY{p}{,} pch\PY{o}{=}\PY{l+m}{22}\PY{p}{,} lty\PY{o}{=}\PY{l+m}{2}\PY{p}{,} col\PY{o}{=}\PY{l+s}{\PYZdq{}}\PY{l+s}{red\PYZdq{}}\PY{p}{)}
         legend\PY{p}{(}\PY{l+s}{\PYZdq{}}\PY{l+s}{topright\PYZdq{}}\PY{p}{,} inset\PY{o}{=}\PY{l+m}{.05}\PY{p}{,} \PY{k+kt}{c}\PY{p}{(}\PY{l+s}{\PYZdq{}}\PY{l+s}{Apple\PYZdq{}}\PY{p}{,}\PY{l+s}{\PYZdq{}}\PY{l+s}{Facebook\PYZdq{}}\PY{p}{)}\PY{p}{,} fill\PY{o}{=}\PY{k+kt}{c}\PY{p}{(}\PY{l+s}{\PYZdq{}}\PY{l+s}{blue\PYZdq{}}\PY{p}{,}\PY{l+s}{\PYZdq{}}\PY{l+s}{red\PYZdq{}}\PY{p}{)}\PY{p}{,} horiz\PY{o}{=}\PY{k+kc}{TRUE}\PY{p}{)}
         hist\PY{p}{(}aapl1\PY{o}{\PYZdl{}}Adj.Close\PY{p}{,} col\PY{o}{=}rainbow\PY{p}{(}\PY{l+m}{8}\PY{p}{)}\PY{p}{)}
\end{Verbatim}


    \begin{center}
    \adjustimage{max size={0.9\linewidth}{0.9\paperheight}}{output_44_0.png}
    \end{center}
    { \hspace*{\fill} \\}
    
    \begin{center}
    \adjustimage{max size={0.9\linewidth}{0.9\paperheight}}{output_44_1.png}
    \end{center}
    { \hspace*{\fill} \\}
    
    \subsection{Problem 5}\label{problem-5}

    \begin{Verbatim}[commandchars=\\\{\}]
{\color{incolor}In [{\color{incolor}29}]:} \PY{c+c1}{\PYZsh{} Displays all datasets.}
         data\PY{p}{(}\PY{p}{)}
\end{Verbatim}


    \begin{Verbatim}[commandchars=\\\{\}]
{\color{incolor}In [{\color{incolor}30}]:} \PY{k+kp}{head}\PY{p}{(}women\PY{p}{)}
\end{Verbatim}


    \begin{tabular}{r|ll}
 height & weight\\
\hline
	 58  & 115\\
	 59  & 117\\
	 60  & 120\\
	 61  & 123\\
	 62  & 126\\
	 63  & 129\\
\end{tabular}


    
    \begin{Verbatim}[commandchars=\\\{\}]
{\color{incolor}In [{\color{incolor}31}]:} \PY{k+kp}{summary}\PY{p}{(}women\PY{p}{)}
\end{Verbatim}


    
    \begin{verbatim}
     height         weight     
 Min.   :58.0   Min.   :115.0  
 1st Qu.:61.5   1st Qu.:124.5  
 Median :65.0   Median :135.0  
 Mean   :65.0   Mean   :136.7  
 3rd Qu.:68.5   3rd Qu.:148.0  
 Max.   :72.0   Max.   :164.0  
    \end{verbatim}

    
    \begin{Verbatim}[commandchars=\\\{\}]
{\color{incolor}In [{\color{incolor}32}]:} plot\PY{p}{(}women\PY{p}{)}
\end{Verbatim}


    \begin{center}
    \adjustimage{max size={0.9\linewidth}{0.9\paperheight}}{output_49_0.png}
    \end{center}
    { \hspace*{\fill} \\}
    
    \begin{Verbatim}[commandchars=\\\{\}]
{\color{incolor}In [{\color{incolor}33}]:} \PY{k+kp}{head}\PY{p}{(}uspop\PY{p}{)}
         plot\PY{p}{(}uspop\PY{p}{)}
\end{Verbatim}


    \begin{enumerate*}
\item 3.93
\item 5.31
\item 7.24
\item 9.64
\item 12.9
\item 17.1
\end{enumerate*}


    
    \begin{center}
    \adjustimage{max size={0.9\linewidth}{0.9\paperheight}}{output_50_1.png}
    \end{center}
    { \hspace*{\fill} \\}
    
    \subsection{Problem 6}\label{problem-6}

    \begin{Verbatim}[commandchars=\\\{\}]
{\color{incolor}In [{\color{incolor}34}]:} \PY{c+c1}{\PYZsh{} Use libraries ggmap, maptools.}
         \PY{c+c1}{\PYZsh{} register\PYZus{}google(Key = \PYZsq{}\PYZsq{}) \PYZhy{}\PYZgt{} API Key is given to access API.}
         \PY{c+c1}{\PYZsh{} All the places to point on map are given into a vector named \PYZsq{}visited\PYZsq{}.}
         \PY{c+c1}{\PYZsh{} Get Latitude and Longitude of each place and plot it on map.}
         
         \PY{k+kn}{library}\PY{p}{(}\PY{l+s}{\PYZdq{}}\PY{l+s}{ggmap\PYZdq{}}\PY{p}{)}
         \PY{k+kn}{library}\PY{p}{(}\PY{l+s}{\PYZdq{}}\PY{l+s}{maptools\PYZdq{}}\PY{p}{)}
         \PY{k+kn}{library}\PY{p}{(}maps\PY{p}{)}
         
         \PY{c+c1}{\PYZsh{} Please provide API. }
         register\PYZus{}google\PY{p}{(}key \PY{o}{=} \PY{l+s}{\PYZdq{}}\PY{l+s}{\PYZdq{}}\PY{p}{)} 
         visited \PY{o}{\PYZlt{}\PYZhy{}} \PY{k+kt}{c}\PY{p}{(}\PY{l+s}{\PYZdq{}}\PY{l+s}{SFO\PYZdq{}}\PY{p}{,} \PY{l+s}{\PYZdq{}}\PY{l+s}{Chennai\PYZdq{}}\PY{p}{,} \PY{l+s}{\PYZdq{}}\PY{l+s}{London\PYZdq{}}\PY{p}{,} \PY{l+s}{\PYZdq{}}\PY{l+s}{Melbourne\PYZdq{}}\PY{p}{,}\PY{l+s}{\PYZdq{}}\PY{l+s}{Lima,Peru\PYZdq{}}\PY{p}{,} \PY{l+s}{\PYZdq{}}\PY{l+s}{Johannesbury, SA\PYZdq{}}\PY{p}{)}
         ll.visited \PY{o}{\PYZlt{}\PYZhy{}} geocode\PY{p}{(}visited\PY{p}{)}
         visit.x \PY{o}{\PYZlt{}\PYZhy{}} ll.visited\PY{o}{\PYZdl{}}lon
         visit.y \PY{o}{\PYZlt{}\PYZhy{}} ll.visited\PY{o}{\PYZdl{}}lat
         map\PY{p}{(}\PY{l+s}{\PYZdq{}}\PY{l+s}{world\PYZdq{}}\PY{p}{,} fill\PY{o}{=}\PY{k+kc}{TRUE}\PY{p}{,} col\PY{o}{=}\PY{l+s}{\PYZdq{}}\PY{l+s}{white\PYZdq{}}\PY{p}{,} bg\PY{o}{=}\PY{l+s}{\PYZdq{}}\PY{l+s}{lightblue\PYZdq{}}\PY{p}{,} ylim\PY{o}{=}\PY{k+kt}{c}\PY{p}{(}\PY{l+m}{\PYZhy{}60}\PY{p}{,} \PY{l+m}{90}\PY{p}{)}\PY{p}{,} mar\PY{o}{=}\PY{k+kt}{c}\PY{p}{(}\PY{l+m}{0}\PY{p}{,}\PY{l+m}{0}\PY{p}{,}\PY{l+m}{0}\PY{p}{,}\PY{l+m}{0}\PY{p}{)}\PY{p}{)}
         points\PY{p}{(}visit.x\PY{p}{,}visit.y\PY{p}{,} col\PY{o}{=}\PY{l+s}{\PYZdq{}}\PY{l+s}{red\PYZdq{}}\PY{p}{,} pch\PY{o}{=}\PY{l+m}{36}\PY{p}{)}
\end{Verbatim}


    \begin{Verbatim}[commandchars=\\\{\}]
Loading required package: ggplot2
Google's Terms of Service: https://cloud.google.com/maps-platform/terms/.
Please cite ggmap if you use it! See citation("ggmap") for details.
Loading required package: sp
Checking rgeos availability: FALSE
 	Note: when rgeos is not available, polygon geometry 	computations in maptools depend on gpclib,
 	which has a restricted licence. It is disabled by default;
 	to enable gpclib, type gpclibPermit()
Source : https://maps.googleapis.com/maps/api/geocode/json?address=SFO\&key=xxx-Sm541do
Source : https://maps.googleapis.com/maps/api/geocode/json?address=Chennai\&key=xxx-Sm541do
Source : https://maps.googleapis.com/maps/api/geocode/json?address=London\&key=xxx-Sm541do
Source : https://maps.googleapis.com/maps/api/geocode/json?address=Melbourne\&key=xxx-Sm541do
Source : https://maps.googleapis.com/maps/api/geocode/json?address=Lima,Peru\&key=xxx-Sm541do
Source : https://maps.googleapis.com/maps/api/geocode/json?address=Johannesbury,+SA\&key=xxx-Sm541do

    \end{Verbatim}

    \begin{center}
    \adjustimage{max size={0.9\linewidth}{0.9\paperheight}}{output_52_1.png}
    \end{center}
    { \hspace*{\fill} \\}
    
    \begin{Verbatim}[commandchars=\\\{\}]
{\color{incolor}In [{\color{incolor}39}]:} \PY{c+c1}{\PYZsh{} Use libraries ggmap, maptools.}
         \PY{c+c1}{\PYZsh{} All the places to point on map are given into a vector named \PYZsq{}visited\PYZsq{}.}
         \PY{c+c1}{\PYZsh{} Get Latitude and Longitude of each place and plot it on map.}
         
         \PY{k+kn}{library}\PY{p}{(}\PY{l+s}{\PYZdq{}}\PY{l+s}{ggmap\PYZdq{}}\PY{p}{)}
         \PY{k+kn}{library}\PY{p}{(}\PY{l+s}{\PYZdq{}}\PY{l+s}{maptools\PYZdq{}}\PY{p}{)}
         \PY{k+kn}{library}\PY{p}{(}maps\PY{p}{)}
         visited \PY{o}{\PYZlt{}\PYZhy{}} \PY{k+kt}{c}\PY{p}{(}\PY{l+s}{\PYZdq{}}\PY{l+s}{SFO\PYZdq{}}\PY{p}{,} \PY{l+s}{\PYZdq{}}\PY{l+s}{New York\PYZdq{}}\PY{p}{,} \PY{l+s}{\PYZdq{}}\PY{l+s}{Buffalo\PYZdq{}}\PY{p}{,} \PY{l+s}{\PYZdq{}}\PY{l+s}{Dallas, TX\PYZdq{}}\PY{p}{)}
         ll.visited \PY{o}{\PYZlt{}\PYZhy{}} geocode\PY{p}{(}visited\PY{p}{)}
         visit.x \PY{o}{\PYZlt{}\PYZhy{}} ll.visited\PY{o}{\PYZdl{}}lon
         visit.y \PY{o}{\PYZlt{}\PYZhy{}} ll.visited\PY{o}{\PYZdl{}}lat
         map\PY{p}{(}\PY{l+s}{\PYZdq{}}\PY{l+s}{state\PYZdq{}}\PY{p}{,} fill\PY{o}{=}\PY{k+kc}{TRUE}\PY{p}{,} col\PY{o}{=}rainbow\PY{p}{(}\PY{l+m}{50}\PY{p}{)}\PY{p}{,} bg\PY{o}{=}\PY{l+s}{\PYZdq{}}\PY{l+s}{lightblue\PYZdq{}}\PY{p}{,} mar\PY{o}{=}\PY{k+kt}{c}\PY{p}{(}\PY{l+m}{0}\PY{p}{,}\PY{l+m}{0}\PY{p}{,}\PY{l+m}{0}\PY{p}{,}\PY{l+m}{0}\PY{p}{)}\PY{p}{)}
         points\PY{p}{(}visit.x\PY{p}{,}visit.y\PY{p}{,} col\PY{o}{=}\PY{l+s}{\PYZdq{}}\PY{l+s}{yellow\PYZdq{}}\PY{p}{,} pch\PY{o}{=}\PY{l+m}{36}\PY{p}{)}
\end{Verbatim}


    \begin{Verbatim}[commandchars=\\\{\}]
Source : https://maps.googleapis.com/maps/api/geocode/json?address=SFO\&key=xxx-Sm541do
Source : https://maps.googleapis.com/maps/api/geocode/json?address=New+York\&key=xxx-Sm541do
Source : https://maps.googleapis.com/maps/api/geocode/json?address=Buffalo\&key=xxx-Sm541do
Source : https://maps.googleapis.com/maps/api/geocode/json?address=Dallas,+TX\&key=xxx-Sm541do

    \end{Verbatim}

    \begin{center}
    \adjustimage{max size={0.9\linewidth}{0.9\paperheight}}{output_53_1.png}
    \end{center}
    { \hspace*{\fill} \\}
    
    \subsection{Problem 7}\label{problem-7}

    \begin{Verbatim}[commandchars=\\\{\}]
{\color{incolor}In [{\color{incolor}42}]:} \PY{k+kn}{attach}\PY{p}{(}mtcars\PY{p}{)}
         \PY{k+kp}{head}\PY{p}{(}mtcars\PY{p}{)}
         plot\PY{p}{(}mtcars\PY{p}{[}\PY{k+kt}{c}\PY{p}{(}\PY{l+m}{1}\PY{p}{,}\PY{l+m}{3}\PY{p}{,}\PY{l+m}{4}\PY{p}{,}\PY{l+m}{5}\PY{p}{,}\PY{l+m}{6}\PY{p}{)}\PY{p}{]}\PY{p}{,} main\PY{o}{=}\PY{l+s}{\PYZdq{}}\PY{l+s}{MTCARS Data\PYZdq{}}\PY{p}{)}
         plot\PY{p}{(}mtcars\PY{p}{[}\PY{k+kt}{c}\PY{p}{(}\PY{l+m}{1}\PY{p}{,}\PY{l+m}{3}\PY{p}{,}\PY{l+m}{4}\PY{p}{,}\PY{l+m}{6}\PY{p}{)}\PY{p}{]}\PY{p}{,} main\PY{o}{=}\PY{l+s}{\PYZdq{}}\PY{l+s}{MTCARS Data\PYZdq{}}\PY{p}{)}
         plot\PY{p}{(}mtcars\PY{p}{[}\PY{k+kt}{c}\PY{p}{(}\PY{l+m}{1}\PY{p}{,}\PY{l+m}{3}\PY{p}{,}\PY{l+m}{4}\PY{p}{,}\PY{l+m}{6}\PY{p}{)}\PY{p}{]}\PY{p}{,} col\PY{o}{=}rainbow\PY{p}{(}\PY{l+m}{5}\PY{p}{)}\PY{p}{,}main\PY{o}{=}\PY{l+s}{\PYZdq{}}\PY{l+s}{MTCARS Data\PYZdq{}}\PY{p}{)}
\end{Verbatim}


    \begin{Verbatim}[commandchars=\\\{\}]
The following object is masked from package:ggplot2:

    mpg


    \end{Verbatim}

    \begin{tabular}{r|lllllllllll}
  & mpg & cyl & disp & hp & drat & wt & qsec & vs & am & gear & carb\\
\hline
	Mazda RX4 & 21.0  & 6     & 160   & 110   & 3.90  & 2.620 & 16.46 & 0     & 1     & 4     & 4    \\
	Mazda RX4 Wag & 21.0  & 6     & 160   & 110   & 3.90  & 2.875 & 17.02 & 0     & 1     & 4     & 4    \\
	Datsun 710 & 22.8  & 4     & 108   &  93   & 3.85  & 2.320 & 18.61 & 1     & 1     & 4     & 1    \\
	Hornet 4 Drive & 21.4  & 6     & 258   & 110   & 3.08  & 3.215 & 19.44 & 1     & 0     & 3     & 1    \\
	Hornet Sportabout & 18.7  & 8     & 360   & 175   & 3.15  & 3.440 & 17.02 & 0     & 0     & 3     & 2    \\
	Valiant & 18.1  & 6     & 225   & 105   & 2.76  & 3.460 & 20.22 & 1     & 0     & 3     & 1    \\
\end{tabular}


    
    \begin{center}
    \adjustimage{max size={0.9\linewidth}{0.9\paperheight}}{output_55_2.png}
    \end{center}
    { \hspace*{\fill} \\}
    
    \begin{center}
    \adjustimage{max size={0.9\linewidth}{0.9\paperheight}}{output_55_3.png}
    \end{center}
    { \hspace*{\fill} \\}
    
    \begin{center}
    \adjustimage{max size={0.9\linewidth}{0.9\paperheight}}{output_55_4.png}
    \end{center}
    { \hspace*{\fill} \\}
    
    \begin{Verbatim}[commandchars=\\\{\}]
{\color{incolor}In [{\color{incolor}43}]:} \PY{k+kp}{head}\PY{p}{(}WWWusage\PY{p}{)}
         plot\PY{p}{(}WWWusage\PY{p}{,}type\PY{o}{=}\PY{l+s}{\PYZdq{}}\PY{l+s}{o\PYZdq{}}\PY{p}{,}col\PY{o}{=}\PY{l+s}{\PYZdq{}}\PY{l+s}{blue\PYZdq{}}\PY{p}{,}xlab\PY{o}{=}\PY{l+s}{\PYZdq{}}\PY{l+s}{Minute\PYZdq{}}\PY{p}{,}ylab\PY{o}{=}\PY{l+s}{\PYZdq{}}\PY{l+s}{No. of Visitors\PYZdq{}}\PY{p}{,}main\PY{o}{=}\PY{l+s}{\PYZdq{}}\PY{l+s}{No. of Visitors to the Serve \PYZbs{}n by Each Minute\PYZdq{}}\PY{p}{)}
\end{Verbatim}


    \begin{enumerate*}
\item 88
\item 84
\item 85
\item 85
\item 84
\item 85
\end{enumerate*}


    
    \begin{center}
    \adjustimage{max size={0.9\linewidth}{0.9\paperheight}}{output_56_1.png}
    \end{center}
    { \hspace*{\fill} \\}
    
    \subsection{Problem 8}\label{problem-8}

    \begin{Verbatim}[commandchars=\\\{\}]
{\color{incolor}In [{\color{incolor}44}]:} \PY{k+kn}{library}\PY{p}{(}ggplot2\PY{p}{)}
         ggplot\PY{p}{(}mtcars\PY{p}{,} aes\PY{p}{(}x\PY{o}{=}mpg\PY{p}{,} y\PY{o}{=}disp\PY{p}{)}\PY{p}{)} \PY{o}{+} geom\PYZus{}point\PY{p}{(}\PY{p}{)}
         
         ggplot\PY{p}{(}mtcars\PY{p}{,} mapping \PY{o}{=} aes\PY{p}{(}x \PY{o}{=} disp\PY{p}{,} y \PY{o}{=} mpg\PY{p}{)}\PY{p}{)} \PY{o}{+} geom\PYZus{}point\PY{p}{(}\PY{p}{)} \PY{o}{+} 
           stat\PYZus{}smooth\PY{p}{(}method \PY{o}{=} \PY{l+s}{\PYZsq{}}\PY{l+s}{lm\PYZsq{}}\PY{p}{)}
         
         ggplot\PY{p}{(}mtcars\PY{p}{,} mapping \PY{o}{=} aes\PY{p}{(}x \PY{o}{=} disp\PY{p}{,} y \PY{o}{=} mpg\PY{p}{,} color \PY{o}{=} \PY{k+kp}{as.factor}\PY{p}{(}cyl\PY{p}{)}\PY{p}{)}\PY{p}{)} \PY{o}{+} geom\PYZus{}point\PY{p}{(}\PY{p}{)}
\end{Verbatim}


    \begin{center}
    \adjustimage{max size={0.9\linewidth}{0.9\paperheight}}{output_58_0.png}
    \end{center}
    { \hspace*{\fill} \\}
    
    \begin{center}
    \adjustimage{max size={0.9\linewidth}{0.9\paperheight}}{output_58_1.png}
    \end{center}
    { \hspace*{\fill} \\}
    
    \begin{center}
    \adjustimage{max size={0.9\linewidth}{0.9\paperheight}}{output_58_2.png}
    \end{center}
    { \hspace*{\fill} \\}
    

    % Add a bibliography block to the postdoc
    
    
    
    \end{document}
